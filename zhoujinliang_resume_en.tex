%% start of file `template.tex'.
%% Copyright 2006-2010 Xavier Danaux (xdanaux@gmail.com).
%
% This work may be distributed and/or modified under the
% conditions of the LaTeX Project Public License version 1.3c,
% available at http://www.latex-project.org/lppl/.


\documentclass[11pt,a4paper]{moderncv}

% moderncv themes
%\moderncvtheme[blue]{casual}                 % optional argument are 'blue' (default), 'orange', 'red', 'green', 'grey' and 'roman' (for roman fonts, instead of sans serif fonts)
\moderncvstyle{classic}                        % 选项参数是 ‘casual’, ‘classic’, ‘oldstyle’ 和 ’banking’
\moderncvcolor{blue}

% character encoding
\usepackage[utf8]{inputenc}                   % replace by the encoding you are using
\usepackage[noindent]{ctex}

% adjust the page margins
\usepackage[scale=0.8]{geometry}
\setlength{\hintscolumnwidth}{3cm}						% if you want to change the width of the column with the dates
%\AtBeginDocument{\setlength{\maketitlenamewidth}{6cm}}  % only for the classic theme, if you want to change the width of your name placeholder (to leave more space for your address details
%\AtBeginDocument{\recomputelengths}                     % required when changes are made to page layout lengths

% personal data
\firstname{Frankee}
\familyname{Zhou}
%\title{Resume title (optional)}               % optional, remove the line if not wanted
\address{}{Xuhui District, Shanghai}    % optional, remove the line if not wanted
\mobile{139-1838-7639}                    % optional, remove the line if not wanted
%\phone{phone (optional)}                      % optional, remove the line if not wanted
%\fax{fax (optional)}                          % optional, remove the line if not wanted
\email{frankee.zhou@gmail.com}                      % optional, remove the line if not wanted
%\homepage{homepage (optional)}                % optional, remove the line if not wanted
\extrainfo{♂ 9/27/1980} % optional, remove the line if not wanted
%\photo[64pt]{sjtu}                         % '64pt' is the height the picture must be resized to and 'picture' is the name of the picture file; optional, remove the line if not wanted
%\quote{Some quote (optional)}                 % optional, remove the line if not wanted

% to show numerical labels in the bibliography; only useful if you make citations in your resume
%\makeatletter
%\renewcommand*{\bibliographyitemlabel}{\@biblabel{\arabic{enumiv}}}
%\makeatother

% bibliography with mutiple entries
%\usepackage{multibib}
%\newcites{book,misc}{{Books},{Others}}

%\nopagenumbers{}                             % uncomment to suppress automatic page numbering for CVs longer than one page
%----------------------------------------------------------------------------------
%            content
%----------------------------------------------------------------------------------
\begin{document}
\maketitle

\section{Enducation}
\cventry{2003.09--2006.03}{Master}{ME of JiaoTong University}{Shanghai}{\textit{(an honor of Excellent Graduate)}}{}  % arguments 3 to 6 can be left empty
\cventry{1999.09--2003.07}{Bachelor}{ME of JiaoTong University}{Shanghai}{}{}

\section{Work Experience}
\subsection{Shanghai Astrob Information Technology Co., Ltd. (2006.03--now)}
\cventry{2011.03--201605}{ALP V2}{}{Technical Manager}{Team Leader}{
ALP V2 is a web map system, like google maps, provide SDK
 of javascript and smart mobile phone, including iphone and Android, and also can connected with REST HTTP api. Subsystem includes Map Tile System, Real-time Tile Rendering system, Poi search/Geocode System, Route System, Traffic System, Auth System,
 the access layer system using LVS, Nginx, Varnish, MysSql, Redis, MongoDb, logging and monitoring system.
\begin{itemize}%
\item Responsible for the ALP V2 system requirements analysis, architecture design.
\item Responsible for developing of the full stack C++ network application framework (similar to Finagle), based on the Cetty, supporting protocals including the protobuf-RPC, REST HTTP.
\item Responsible for developing gearman-based task system, and the distributed real-time map tile rendering system based on this.
\item Responsible for developing POI query system, oriented geographic information, based on the Sphinx (including geographical Chinese characters segmentation system based on the the CSWS).
\item Responsible for developing C++ Authentication \& authentication system, similar to shiro.
\end{itemize}}

\cventry{2008.10--2011.03}{ALP(Astrob LBS Platform)}{}{Technical Manager}{Team Leader}{
ALP is a LBS application platform, focus in GPS navigation, similar to the TSP (Telematics Service Provider), providing real-time traffic information, rich POI, Car Mate Club, call center destination setting, seamless integration to third parties' CTI and SP.
\begin{itemize}%
\item Responsible for the ALP system requirements analysis, architecture design.
\item Lead the team to develop high concurrency socket network framework, based on ACE.
\item Responsible for developing cross-platform (WinCE, Windows, Linux) OSGI like C++ plug-in framework.
\item Responsible for developing C++ persistence layer framework based on ODBC, and RPC framework based on Google Protobuf.
\item Responsible for POI search system based on the Sphinx, and logging system.
\item Using MySql, Redis, MongoDB, Sphinx, Protobuf, the ACE and other open source software.
\item The ALP Solution (including CPND navigation) has been successfully operated in CarSmart project since 2010.
\end{itemize}}

\cventry{2008.10--2011.03}{CPND(Connected PND) GPS Navigation}{}{Technical Manager}{Team Leader}{
CPND navigation software, intergated the function provided by the ALP.
\begin{itemize}
\item Responsible for the requirements analysis and architecture design of the CPND navigation.
\item Responsible for the core C++ library (cross-platform foundation library, logging system, configuration library, the signal slot, etc.).
\item Responsible for the team building.
\item Responsible for developing OSGI plug-in framework, for good scalability, using unit test with gtest.
\item Development under SCRUM, project management using Redmine, continuous integration using Hudson.
\end{itemize}}

\cventry{2007.06--2008.10}{Panasonic pre-installed GPS navigation}{}{Senior Software Engineer}{Team Leader}{
 A new pre-installed GPS navigation software for Taiwan and mainland China market, OEM for Panasonic.
\begin{itemize}%
\item Discuss the requirement with the customer, design the software architecture.
\item Manage the daily project development, lead the teammates to design the framework and develop the module, and control the software quality and the project process.
\item The GPS navigation for Panasonic has launched on the markets of China MainLand and Taiwan.
\end{itemize}}

\cventry{2007.03--2007.06}{GPS navigation customized for GSL}{}{Senior Software Engineer}{Team Leader}{
\begin{itemize}
\item Discuss and confirm the requirement of GPS navigation with the client of GSL, make customization development according to the client’s requirement, take charge integration test and the final release.
\item The GPS navigation for GSL has launched on US market.
\end{itemize}}

\cventry{2006.10--2007.03}{MapManager}{}{Software Engineer}{}{A desktop tool for installing and updating the GPS navigation and map data, supporting multi-language.
\begin{itemize}%
\item Responsible for MapManager requirement analysis, design and development.
\end{itemize}}

\cventry{2006.06--2007.01}{Setting up and improve the process and procedure of the software development}{}{Software Engineer}{}{
\begin{itemize}
\item Responsible for the development of internal software development process (in the assistance of consulting company).
\item The software development process and configuration management applied in company successfully.
\end{itemize}}

\cventry{2006.03--2006.07}{The voice guidance module of the GPS navigation system}{}{Software Engineer}{}{
\begin{itemize}
\item The wave \& ogg player, in Windows CE \& Linux.
\end{itemize}}

\section{Open Source Experience}
\cventry{2010.12--now}{Cetty}{\link[https://github.com/frankee/Cetty]{https://github.com/frankee/Cetty}}{Project Founder}{}{
The Cetty project is a network application framework for easy developing high performance and high scalability servers and clients. It comes from Netty project, and based on boost asio, using proactor network pattern.
}

\section{Research Experience}
\subsection{Robot Research Institute of JiaoTong University (2003.03--2006.02)}
\cventry{2005.08--2006.02}{The tobacco leaf packages unpacking robot system}{}{Project Leader}{}{The robot, cooperated with ABB, can unpack effectively the tobacco packages.
\begin{itemize}%
\item Responsible for the architecture design, and the research of computer vision and control algorithms.
\item The robot system applied to Shanghai Tobacco Company successfully.
\end{itemize}}
\cventry{2003.03--2006.02}{The autonomous mobile robot}{}{Team Member}{}{
Participated in the medium-sized autonomous robot soccer of the RoboCup (the Robot World Cup).
\begin{itemize}%
\item Involved in the robot architecture design, developed the real-time intelligent motion control system based on RTLinux.
\item The competition winner of the 2005 China RoboCup, motion control algorithm published in the Journal of Shanghai JiaoTong University.
\end{itemize}}
\cventry{2004.09--2005.04}{The OMRON entertainment robot}{}{Project Leader}{}{The robot, cooperated with OMRON, can interact with the audience on stage performances.
\begin{itemize}%
\item Responsible for the project development process. Leading robot control software development.
\item Took part in Shanghai International Industry Fair 2004, and interviewed by many news media.
\end{itemize}}
\cventry{2003.03--2004.12}{The intelligent fruit auto-sorting robot}{}{Team Member}{}{Cooperated with the Shanghai Agricultural Committee.
\begin{itemize}%
\item Responsible for the computer vision recognition algorithms development.
\item The rebot got through the Shanghai agricultural committee's approval, and the related research at the magazine of the Robot.
\end{itemize}}

\section{Languages}
\cvlanguage{English}{\small CET-6, Fluent in English}{}

\section{Computer skills}
\cvline{Language}{\small Expert knowledge in C++, familiar with Java, Javascript}
\cvline{Tools}{\small Linux, boost, poco, gtest, Protobuf, ACE, MySql, Redis, MongoDB, Memcached, Sphinx, svn, git, Redmine}
\cvline{Others}{\small UML, OSGI, OOD, SOA, Design Pattern, RUP/SCRUM}

%\section{Interests}
%\cvline{hobby 1}{\small Description}
%\cvline{hobby 2}{\small Description}
%\cvline{hobby 3}{\small Description}

\section{Extra}
\cvlistitem{Open-minded, creativity, responsible, teamwork spirit, good interpersonal skills and leadership.}
%\cvlistitem{Item 2}
%\cvlistitem[+]{Item 3}            % optional other symbol

%\renewcommand{\listitemsymbol}{-} % change the symbol for lists


% Publications from a BibTeX file without multibib\renewcommand*{\bibliographyitemlabel}{\@biblabel{\arabic{enumiv}}}% for BibTeX numerical labels
%\nocite{*}
%\bibliographystyle{plain}
%\bibliography{publications}       % 'publications' is the name of a BibTeX file

% Publications from a BibTeX file using the multibib package
%\section{Publications}
%\nocitebook{book1,book2}
%\bibliographystylebook{plain}
%\bibliographybook{publications}   % 'publications' is the name of a BibTeX file
%\nocitemisc{misc1,misc2,misc3}
%\bibliographystylemisc{plain}
%\bibliographymisc{publications}   % 'publications' is the name of a BibTeX file

\end{document}


%% end of file `template_en.tex'.
